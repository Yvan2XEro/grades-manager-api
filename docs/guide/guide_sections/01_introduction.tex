% Section 01 - Introduction
\section{Introduction}

\subsection{À propos de TKAMS}

\textbf{TKAMS (TK Academic Management System)} est un système de gestion académique complet destiné aux institutions d'enseignement supérieur : facultés, universités et instituts. Cette version 1.0 fournit les fonctionnalités essentielles pour gérer l'ensemble du processus académique, de l'inscription des étudiants jusqu'à la publication des résultats.

\info{Cette version contient les fonctionnalités de base de gestion académique. Des fonctionnalités additionnelles peuvent être développées selon les besoins spécifiques de votre institution.}

Le système offre :

\begin{itemize}[leftmargin=*]
    \item \textbf{Gestion de la structure académique} : Facultés, programmes, cycles d'études, classes et cours
    \item \textbf{Gestion des étudiants} : Inscriptions, parcours académiques, crédits ECTS
    \item \textbf{Gestion des examens} : Planification, types d'examens, sessions, délégations
    \item \textbf{Saisie et validation des notes} : Interface intuitive avec workflows d'approbation
    \item \textbf{Calculs automatiques} : Moyennes, crédits, mentions, résultats finaux
    \item \textbf{Publications et exports} : Relevés de notes, attestations, PV d'examens
    \item \textbf{Gestion des utilisateurs} : Rôles et permissions granulaires
    \item \textbf{Traçabilité complète} : Historique, workflows, notifications
\end{itemize}

L'application est accessible via un navigateur web et fonctionne en mode client-serveur avec une base de données PostgreSQL centralisée.

\subsection{Types d'institutions supportées}

TKAMS est conçu pour s'adapter à différents types d'institutions d'enseignement supérieur :

\begin{description}[leftmargin=3cm, style=nextline]
    \item[Facultés]
    \begin{itemize}
        \item Gestion complète des programmes académiques
        \item Support multi-cycles (Licence, Master, Doctorat)
        \item Exemple : Faculté de Médecine, Faculté de Droit, etc.
    \end{itemize}

    \item[Universités]
    \begin{itemize}
        \item Gestion multi-facultés
        \item Structure hiérarchique complète
        \item Coordination entre départements
    \end{itemize}

    \item[Instituts]
    \begin{itemize}
        \item Programmes spécialisés
        \item Formation professionnelle et technique
        \item Instituts de recherche avec enseignement
    \end{itemize}
\end{description}

\subsection{Portée de cette version}

\begin{tcolorbox}[colback=blue!5!white,colframe=blue!75!black,title=Version 1.0 - Fonctionnalités de base]

\textbf{Cette version inclut :}
\begin{itemize}[leftmargin=*]
    \item Structure académique complète (facultés, programmes, cycles, classes, cours)
    \item Gestion des étudiants et inscriptions
    \item Système de notes et examens
    \item Calcul automatique des moyennes et crédits
    \item Workflows d'approbation
    \item Exports de documents (relevés, attestations)
    \item Système de permissions et sécurité
\end{itemize}

\textbf{Fonctionnalités non incluses (développement sur demande) :}
\begin{itemize}[leftmargin=*]
    \item Gestion financière (frais de scolarité, bourses)
    \item Gestion des emplois du temps
    \item Plateforme e-learning intégrée
    \item Gestion des bibliothèques
    \item Module de recherche et publications
    \item Application mobile
\end{itemize}

\astuce{Contactez le développeur pour ajouter des fonctionnalités spécifiques à votre institution.}
\end{tcolorbox}

\subsection{Architecture technique}

Le système est construit sur une architecture moderne et performante :

\begin{description}[leftmargin=3cm, style=nextline]
    \item[Backend API]
    \begin{itemize}
        \item Runtime : \textbf{Bun} (JavaScript/TypeScript ultra-rapide)
        \item Framework : \textbf{Hono} (serveur HTTP léger et performant)
        \item API : \textbf{tRPC} (type-safety end-to-end entre frontend et backend)
        \item Base de données : \textbf{PostgreSQL} avec \textbf{Drizzle ORM}
        \item Authentification : \textbf{Better-Auth} avec profils de domaine
    \end{itemize}

    \item[Frontend Web]
    \begin{itemize}
        \item Framework : \textbf{React} avec \textbf{Vite}
        \item Routing : \textbf{React Router v7}
        \item UI : \textbf{shadcn/ui} + \textbf{TailwindCSS v4}
        \item Gestion d'état : \textbf{Zustand}
        \item Internationalisation : \textbf{i18next} (Français/Anglais)
    \end{itemize}

    \item[Architecture Monorepo]
    \begin{itemize}
        \item Organisation : Structure Better-T-Stack
        \item \code{apps/server/} : API Backend
        \item \code{apps/web/} : Application frontend
        \item Partage de types TypeScript entre frontend et backend
    \end{itemize}
\end{description}

\subsection{Fonctionnalités principales}

\subsubsection{Gestion de la structure académique}

\begin{description}[leftmargin=3cm, style=nextline]
    \item[Années académiques]
    \begin{itemize}
        \item Gestion des années scolaires (ex: 2024-2025)
        \item Une seule année active à la fois
        \item Bascule entre années académiques
    \end{itemize}

    \item[Hiérarchie organisationnelle]
    \begin{itemize}
        \item \textbf{Facultés} : Unités principales (ex: FMSP)
        \item \textbf{Programmes} : Filières d'études (ex: Médecine, Pharmacie)
        \item \textbf{Cycles d'études} : Niveaux (ex: Licence, Master, Doctorat)
        \item \textbf{Classes} : Groupes d'étudiants (ex: Médecine Année 1)
    \end{itemize}

    \item[Structure pédagogique]
    \begin{itemize}
        \item \textbf{Unités d'Enseignement (UE)} : Regroupements thématiques
        \item \textbf{Cours} : Éléments constitutifs avec coefficients et crédits ECTS
        \item Gestion des semestres (S1, S2)
        \item Configuration flexible des coefficients
    \end{itemize}
\end{description}

\subsubsection{Gestion des étudiants}

\begin{description}[leftmargin=3cm, style=nextline]
    \item[Inscriptions]
    \begin{itemize}
        \item Enregistrement des étudiants dans une classe
        \item Numéros de matricule uniques générés automatiquement
        \item Suivi du parcours académique année par année
    \end{itemize}

    \item[Inscriptions aux cours]
    \begin{itemize}
        \item Association étudiants-cours pour chaque semestre
        \item Gestion des tentatives multiples (redoublement)
        \item Validation automatique selon les règles académiques
    \end{itemize}

    \item[Système de crédits]
    \begin{itemize}
        \item \textbf{Ledger de crédits} : Comptabilité des crédits ECTS
        \item Attribution automatique lors de validation des cours
        \item Suivi cumulatif par étudiant
        \item Conditions de passage (crédits requis)
    \end{itemize}
\end{description}

\subsubsection{Gestion des examens et notes}

\begin{description}[leftmargin=3cm, style=nextline]
    \item[Types d'examens]
    \begin{itemize}
        \item Examens configurables (CC, TP, Examen Final, etc.)
        \item Poids personnalisables dans le calcul de la moyenne
        \item Gestion par semestre
    \end{itemize}

    \item[Planification des examens]
    \begin{itemize}
        \item Programmation automatique des sessions d'examens
        \item Respect des contraintes semestre/cours
        \item Dates limites de saisie configurables
        \item Système de cache pour optimiser les performances
    \end{itemize}

    \item[Délégation de saisie]
    \begin{itemize}
        \item Attribution temporaire de droits de saisie
        \item Gestion par enseignant et par examen
        \item Contrôle des dates de validité
    \end{itemize}

    \item[Saisie des notes]
    \begin{itemize}
        \item Interface conviviale par cours et type d'examen
        \item Validation des notes (0-20)
        \item Calcul automatique des moyennes
        \item Détection des valeurs aberrantes
    \end{itemize}

    \item[Workflows d'approbation]
    \begin{itemize}
        \item Soumission des modifications par les enseignants
        \item Validation hiérarchique (Dean, Administrateur)
        \item Traçabilité complète des changements
        \item Notifications automatiques
    \end{itemize}
\end{description}

\subsubsection{Exports et publications}

\begin{description}[leftmargin=3cm, style=nextline]
    \item[Relevés de notes]
    \begin{itemize}
        \item Génération en HTML (templates Handlebars)
        \item Export PDF via navigateur
        \item Personnalisation des logos et en-têtes
        \item Support multi-semestres
    \end{itemize}

    \item[Attestations de réussite]
    \begin{itemize}
        \item Génération automatique pour étudiants admis
        \item Templates personnalisables
        \item Mentions calculées automatiquement
    \end{itemize}

    \item[PV d'examens]
    \begin{itemize}
        \item Procès-verbaux d'examens officiels
        \item Statistiques de réussite/échec
        \item Signatures électroniques
    \end{itemize}
\end{description}

\subsection{Rôles et permissions}

Le système implémente une gestion avancée des rôles avec hiérarchie et permissions granulaires :

\begin{table}[h]
\centering
\begin{tabular}{|l|p{10cm}|}
\hline
\textbf{Rôle} & \textbf{Permissions principales} \\
\hline
\textbf{Super Admin} &
\begin{itemize}[leftmargin=*,topsep=0pt,partopsep=0pt,itemsep=0pt]
    \item Activation des années académiques
    \item Gestion complète du système
    \item Accès à toutes les fonctionnalités
\end{itemize} \\
\hline
\textbf{Administrator} &
\begin{itemize}[leftmargin=*,topsep=0pt,partopsep=0pt,itemsep=0pt]
    \item Gestion de la structure académique
    \item Validation des workflows de niveau 2
    \item Configuration des examens
    \item Génération des exports
\end{itemize} \\
\hline
\textbf{Dean (Doyen)} &
\begin{itemize}[leftmargin=*,topsep=0pt,partopsep=0pt,itemsep=0pt]
    \item Validation des workflows de niveau 1
    \item Consultation des résultats
    \item Approbation des modifications de notes
\end{itemize} \\
\hline
\textbf{Teacher (Enseignant)} &
\begin{itemize}[leftmargin=*,topsep=0pt,partopsep=0pt,itemsep=0pt]
    \item Saisie des notes pour ses cours
    \item Soumission de demandes de modification
    \item Consultation des résultats de ses cours
\end{itemize} \\
\hline
\textbf{Staff (Personnel)} &
\begin{itemize}[leftmargin=*,topsep=0pt,partopsep=0pt,itemsep=0pt]
    \item Consultation limitée
    \item Assistance administrative
\end{itemize} \\
\hline
\textbf{Student (Étudiant)} &
\begin{itemize}[leftmargin=*,topsep=0pt,partopsep=0pt,itemsep=0pt]
    \item Consultation de ses propres résultats
    \item Téléchargement de ses relevés de notes
\end{itemize} \\
\hline
\end{tabular}
\caption{Rôles et permissions du système}
\end{table}

\info{Les rôles sont hiérarchiques : un Super Admin peut faire tout ce qu'un Administrator peut faire, un Administrator peut faire tout ce qu'un Dean peut faire, etc.}

\subsection{Organisation des modules}

L'interface de l'application est organisée en modules fonctionnels :

\subsubsection{Modules académiques}

\begin{enumerate}[leftmargin=*]
    \item \menu{Années académiques} - Gestion des années scolaires
    \item \menu{Facultés} - Administration des facultés
    \item \menu{Programmes} - Gestion des filières d'études
    \item \menu{Cycles d'études} - Configuration des niveaux (L, M, D)
    \item \menu{Classes} - Gestion des groupes d'étudiants
    \item \menu{Semestres} - Configuration des périodes académiques
    \item \menu{Unités d'enseignement} - Gestion des UE
    \item \menu{Cours} - Configuration des éléments constitutifs
\end{enumerate}

\subsubsection{Modules étudiants}

\begin{enumerate}[leftmargin=*]
    \item \menu{Étudiants} - Base de données étudiants
    \item \menu{Inscriptions} - Gestion des inscriptions annuelles
    \item \menu{Inscriptions aux cours} - Affectation cours/étudiants
    \item \menu{Ledger de crédits} - Suivi des crédits ECTS
\end{enumerate}

\subsubsection{Modules examens et notes}

\begin{enumerate}[leftmargin=*]
    \item \menu{Types d'examens} - Configuration des types d'évaluation
    \item \menu{Planification des examens} - Programmation des sessions
    \item \menu{Délégations} - Attribution des droits de saisie
    \item \menu{Saisie des notes} - Interface de saisie
    \item \menu{Notes} - Consultation et gestion
\end{enumerate}

\subsubsection{Modules workflow et administration}

\begin{enumerate}[leftmargin=*]
    \item \menu{Workflows} - Gestion des demandes d'approbation
    \item \menu{Notifications} - Système de notifications
    \item \menu{Utilisateurs} - Gestion des comptes utilisateurs
    \item \menu{Profils de domaine} - Profils métier des utilisateurs
    \item \menu{Exports} - Génération de documents (relevés, attestations)
\end{enumerate}

\subsection{Configuration requise}

\begin{tcolorbox}[colback=blue!5!white,colframe=blue!75!black,title=Configuration système minimale]

\textbf{Serveur :}
\begin{itemize}[leftmargin=*]
    \item \textbf{Système d'exploitation} : Linux (Ubuntu 20.04+), Windows Server 2019+, macOS
    \item \textbf{Runtime} : Bun v1.0+ (ou Node.js v20+ en fallback)
    \item \textbf{Base de données} : PostgreSQL 14+
    \item \textbf{Mémoire RAM} : 2 Go minimum (4 Go recommandé)
    \item \textbf{Espace disque} : 1 Go pour l'application + données
    \item \textbf{Processeur} : 2 cœurs minimum
\end{itemize}

\textbf{Clients (navigateur web) :}
\begin{itemize}[leftmargin=*]
    \item \textbf{Navigateurs supportés} : Chrome 90+, Firefox 88+, Safari 14+, Edge 90+
    \item \textbf{Résolution écran} : 1280x720 minimum (1920x1080 recommandé)
    \item \textbf{Connexion Internet} : Haut débit recommandé
    \item \textbf{JavaScript} : Activé et à jour
\end{itemize}
\end{tcolorbox}

\subsection{Sécurité et confidentialité}

\begin{tcolorbox}[colback=yellow!10!white,colframe=orange!75!black,title=Mesures de sécurité]

\textbf{Authentification et autorisation :}
\begin{itemize}[leftmargin=*]
    \item \textbf{Better-Auth} : Système d'authentification robuste
    \item \textbf{Sessions sécurisées} : Cookies HTTP-only, SameSite
    \item \textbf{Hachage des mots de passe} : bcrypt avec salt
    \item \textbf{Profils séparés} : Séparation auth/business logic
    \item \textbf{RBAC (Role-Based Access Control)} : Contrôle d'accès granulaire
\end{itemize}

\textbf{Protection des données :}
\begin{itemize}[leftmargin=*]
    \item \textbf{HTTPS recommandé} : Chiffrement des communications
    \item \textbf{Validation des entrées} : Zod schemas pour toutes les données
    \item \textbf{Prévention SQL injection} : ORM Drizzle avec requêtes paramétrées
    \item \textbf{CORS configuré} : Contrôle des origines autorisées
    \item \textbf{Traçabilité complète} : Logs et audit trails
\end{itemize}

\textbf{Recommandations de déploiement :}
\begin{itemize}[leftmargin=*]
    \item Utiliser HTTPS en production (certificat SSL/TLS)
    \item Configurer un pare-feu pour limiter l'accès
    \item Effectuer des sauvegardes régulières de la base de données
    \item Mettre à jour régulièrement les dépendances
    \item Utiliser des mots de passe forts pour tous les comptes
    \item Activer l'authentification à deux facteurs (si disponible)
\end{itemize}
\end{tcolorbox}

\subsection{Support et assistance}

\begin{description}[leftmargin=3cm, style=nextline]
    \item[Documentation]
    \begin{itemize}
        \item Guide utilisateur (ce document)
        \item Documentation technique (CLAUDE.md)
        \item README du projet
    \end{itemize}

    \item[Contact]
    \begin{itemize}
        \item Email : cedrictefoye@gmail.com
        \item Support technique : Ouvert aux heures ouvrables
    \end{itemize}

    \item[Mises à jour]
    \begin{itemize}
        \item Vérification des mises à jour recommandée
        \item Notes de version disponibles
        \item Migration assistée entre versions
    \end{itemize}
\end{description}

\newpage
