% Section 05 - Gestion des notes et examens
\section{Gestion des notes et examens}

\subsection{Types d'examens}

\subsubsection{Configuration des types}

Les types d'examens définissent les modalités d'évaluation :

\begin{table}[h]
\centering
\begin{tabular}{|l|l|c|}
\hline
\textbf{Type} & \textbf{Description} & \textbf{Poids} \\
\hline
CC (Contrôle Continu) & Évaluations régulières & 30\% \\
TP (Travaux Pratiques) & Évaluations pratiques & 20\% \\
Examen Final & Examen de fin de semestre & 50\% \\
\hline
\end{tabular}
\caption{Exemple de types d'examens}
\end{table}

\begin{enumerate}[leftmargin=*]
    \item Menu \menu{Types d'examens} > \bouton{Nouveau type}
    \item Remplissez :
    \begin{itemize}
        \item \champ{Code} : Ex. "CC", "EF", "TP"
        \item \champ{Nom} : Ex. "Contrôle Continu"
        \item \champ{Poids} : Pourcentage dans la moyenne (ex: 30 pour 30\%)
        \item \champ{Description} : Optionnel
    \end{itemize}
    \item \bouton{Créer}
\end{enumerate}

\info{La somme des poids de tous les types d'examens d'un cours doit faire 100\%.}

\subsection{Planification des examens}

\subsubsection{Création manuelle d'un examen}

\begin{enumerate}[leftmargin=*]
    \item Menu \menu{Examens} > \bouton{Nouvel examen}
    \item Sélectionnez :
    \begin{itemize}
        \item \champ{Cours}
        \item \champ{Type d'examen}
        \item \champ{Semestre}
    \end{itemize}
    \item Configurez les dates :
    \begin{itemize}
        \item \champ{Date de l'examen} : Date de passation
        \item \champ{Date limite de saisie} : Deadline pour entrer les notes
        \item \champ{Date de clôture automatique} : Fermeture automatique (optionnel)
    \end{itemize}
    \item \bouton{Créer}
\end{enumerate}

\subsubsection{Planification automatique (Exam Scheduler)}

Pour générer tous les examens d'un coup :

\begin{enumerate}[leftmargin=*]
    \item Menu \menu{Planification des examens}
    \item Sélectionnez :
    \begin{itemize}
        \item \champ{Semestre} : S1 ou S2
        \item \champ{Classe} : Classe concernée (ou toutes)
    \end{itemize}
    \item Le système génère automatiquement :
    \begin{itemize}
        \item Un examen pour chaque type configuré
        \item Pour chaque cours du semestre
        \item Avec des dates par défaut
    \end{itemize}
    \item Vérifiez et ajustez les dates si nécessaire
    \item \bouton{Valider la planification}
\end{enumerate}

\begin{figure}[H]
    \centering
    \fbox{\parbox{0.9\textwidth}{\centering\vspace{2cm}[Capture : Interface de planification automatique]\\\vspace{2cm}}}
    \caption{Planification automatique des examens}
    \label{fig:exam-scheduling}
\end{figure}

\subsection{Délégation de saisie}

Le système permet de déléguer temporairement la saisie des notes aux enseignants.

\subsubsection{Création d'une délégation}

\begin{enumerate}[leftmargin=*]
    \item Menu \menu{Délégations} > \bouton{Nouvelle délégation}
    \item Sélectionnez :
    \begin{itemize}
        \item \champ{Enseignant} : Utilisateur bénéficiaire
        \item \champ{Examen} : Examen concerné
        \item \champ{Date de début} : Début de validité
        \item \champ{Date de fin} : Expiration de la délégation
    \end{itemize}
    \item \bouton{Créer}
\end{enumerate}

\info{Une délégation expire automatiquement après la date de fin. L'enseignant ne peut plus saisir de notes après expiration.}

\subsubsection{Gestion des délégations}

Les administrateurs peuvent :
\begin{itemize}[leftmargin=*]
    \item Consulter toutes les délégations actives
    \item Révoquer une délégation avant son expiration
    \item Prolonger une délégation
    \item Voir l'historique des délégations
\end{itemize}

\subsection{Saisie des notes}

\subsubsection{Interface de saisie}

\begin{enumerate}[leftmargin=*]
    \item Menu \menu{Saisie des notes}
    \item Sélectionnez :
    \begin{itemize}
        \item \champ{Cours}
        \item \champ{Type d'examen}
        \item \champ{Semestre}
    \end{itemize}
    \item L'interface affiche :
    \begin{itemize}
        \item Liste des étudiants inscrits au cours
        \item Champ de saisie pour chaque étudiant
        \item Statistiques en temps réel
    \end{itemize}
\end{enumerate}

\subsubsection{Saisie des notes}

\begin{enumerate}[leftmargin=*]
    \item Entrez la note de chaque étudiant (0 à 20)
    \item Les notes sont validées automatiquement :
    \begin{itemize}
        \item Doivent être entre 0 et 20
        \item Format accepté : entiers ou décimaux (ex: 15.5)
    \end{itemize}
    \item Marquez les absences si nécessaire (checkbox "Absent")
    \item Cliquez sur \bouton{Enregistrer les notes}
\end{enumerate}

\begin{figure}[H]
    \centering
    \fbox{\parbox{0.9\textwidth}{\centering\vspace{2cm}[Capture : Interface de saisie des notes]\\\vspace{2cm}}}
    \caption{Interface de saisie des notes}
    \label{fig:grade-entry}
\end{figure}

\subsubsection{Import depuis Excel}

Pour saisir rapidement de nombreuses notes :

\begin{enumerate}[leftmargin=*]
    \item Dans l'interface de saisie, cliquez sur \bouton{Importer depuis Excel}
    \item Téléchargez le modèle pré-rempli (contient déjà la liste des étudiants)
    \item Remplissez les notes dans le fichier Excel
    \item Importez le fichier
    \item Vérifiez le rapport de validation
    \item \bouton{Confirmer l'import}
\end{enumerate}

\subsection{Calcul des moyennes}

\subsubsection{Moyenne d'un cours}

La moyenne d'un cours est calculée automatiquement selon :

\begin{equation}
Moyenne\_Cours = \sum_{i=1}^{n} (Note_i \times Poids_i)
\end{equation}

Où :
\begin{itemize}[leftmargin=*]
    \item $Note_i$ : Note obtenue au type d'examen $i$
    \item $Poids_i$ : Poids du type d'examen $i$ (en \%)
    \item $n$ : Nombre de types d'examens pour ce cours
\end{itemize}

\textbf{Exemple :}
\begin{itemize}[leftmargin=*]
    \item CC = 12/20 (poids 30\%)
    \item TP = 14/20 (poids 20\%)
    \item Examen Final = 16/20 (poids 50\%)
    \item Moyenne = (12 × 0.30) + (14 × 0.20) + (16 × 0.50) = 14.4/20
\end{itemize}

\subsubsection{Moyenne de l'UE}

La moyenne d'une UE est calculée selon :

\begin{equation}
Moyenne\_UE = \frac{\sum_{j=1}^{m} (Moyenne\_Cours_j \times Coef_j)}{\sum_{j=1}^{m} Coef_j}
\end{equation}

Où :
\begin{itemize}[leftmargin=*]
    \item $Moyenne\_Cours_j$ : Moyenne du cours $j$
    \item $Coef_j$ : Coefficient du cours $j$
    \item $m$ : Nombre de cours dans l'UE
\end{itemize}

\subsubsection{Moyenne générale}

La moyenne générale du semestre est calculée selon :

\begin{equation}
Moyenne\_Generale = \frac{\sum_{k=1}^{p} Moyenne\_UE_k}{p}
\end{equation}

Ou avec pondération si les UE ont des coefficients différents.

\subsection{Validation et publication}

\subsubsection{Validation des notes}

Avant publication, les notes doivent être validées :

\begin{enumerate}[leftmargin=*]
    \item L'enseignant saisit les notes
    \item L'enseignant soumet pour validation
    \item Le Dean (Doyen) vérifie et approuve
    \item L'Administrateur fait la validation finale
    \item Les notes sont alors publiées aux étudiants
\end{enumerate}

\subsubsection{Workflow d'approbation}

Le système utilise un workflow à deux niveaux :

\begin{itemize}[leftmargin=*]
    \item \textbf{Niveau 1} : Validation par le Dean
    \item \textbf{Niveau 2} : Validation finale par l'Administrateur
\end{itemize}

Chaque niveau peut approuver ou rejeter avec un commentaire.

\subsection{Modification des notes}

\subsubsection{Demande de modification}

Une fois les notes validées, toute modification nécessite un workflow :

\begin{enumerate}[leftmargin=*]
    \item L'enseignant identifie une erreur
    \item Menu \menu{Workflows} > \bouton{Nouvelle demande}
    \item Sélectionne la note à modifier
    \item Indique la nouvelle valeur
    \item Justifie la modification
    \item Soumet la demande
\end{enumerate}

\subsubsection{Traitement de la demande}

\begin{enumerate}[leftmargin=*]
    \item Le Dean reçoit une notification
    \item Examine la demande
    \item Approuve ou rejette (niveau 1)
    \item Si approuvé niveau 1 : passe à l'Administrateur
    \item L'Administrateur fait la validation finale (niveau 2)
    \item Si approuvé : la note est mise à jour automatiquement
\end{enumerate}

\begin{figure}[H]
    \centering
    \fbox{\parbox{0.9\textwidth}{\centering\vspace{2cm}[Capture : Interface de workflow]\\\vspace{2cm}}}
    \caption{Gestion des workflows d'approbation}
    \label{fig:workflow}
\end{figure}

\subsection{Statistiques et analyses}

\subsubsection{Statistiques par cours}

Pour chaque cours, visualisez :
\begin{itemize}[leftmargin=*]
    \item Moyenne de la classe
    \item Note minimale et maximale
    \item Écart-type
    \item Taux de réussite ($\geq$ 10/20)
    \item Distribution des notes (histogramme)
\end{itemize}

\subsubsection{Statistiques par classe}

\begin{itemize}[leftmargin=*]
    \item Moyenne générale de la classe
    \item Nombre d'admis / ajournés
    \item Répartition des mentions
    \item Classement des étudiants
\end{itemize}

\subsection{Bonnes pratiques}

\begin{tcolorbox}[colback=green!5!white,colframe=green!75!black,title=Recommandations pour la saisie des notes]

\textbf{Préparation :}
\begin{itemize}[leftmargin=*]
    \item Vérifiez que tous les étudiants sont bien inscrits au cours
    \item Assurez-vous que les types d'examens sont configurés
    \item Planifiez les examens avant la période de saisie
\end{itemize}

\textbf{Saisie :}
\begin{itemize}[leftmargin=*]
    \item Saisissez les notes rapidement après les examens
    \item Vérifiez les valeurs aberrantes
    \item Utilisez l'import Excel pour gagner du temps
    \item Marquez clairement les absences
\end{itemize}

\textbf{Validation :}
\begin{itemize}[leftmargin=*]
    \item Relisez avant de soumettre pour validation
    \item Documentez toute modification demandée
    \item Respectez les délais de validation
\end{itemize}

\textbf{Sécurité :}
\begin{itemize}[leftmargin=*]
    \item Ne partagez jamais vos identifiants de saisie
    \item Déconnectez-vous après chaque session
    \item Sauvegardez vos fichiers Excel source
\end{itemize}
\end{tcolorbox}

\newpage
