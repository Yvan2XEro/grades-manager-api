% Section 11 - Cas d'usage pratiques
\section{Cas d'usage pratiques}

\subsection{Scénario 1 : Début d'année académique}

\subsubsection{Contexte}

Vous êtes administrateur et devez préparer le système pour une nouvelle année académique 2024-2025.

\subsubsection{Étapes}

\begin{enumerate}[leftmargin=*]
    \item \textbf{Créer l'année académique}
    \begin{itemize}
        \item Menu \menu{Années académiques} > \bouton{Nouvelle année}
        \item Année : 2024-2025
        \item Dates : 01/09/2024 au 31/08/2025
        \item \bouton{Activer} (Super Admin uniquement)
    \end{itemize}

    \item \textbf{Vérifier la structure académique}
    \begin{itemize}
        \item Facultés, programmes, cycles : à jour ?
        \item Classes : créer les nouvelles classes pour l'année
        \item Cours : vérifier que tous les cours sont configurés
    \end{itemize}

    \item \textbf{Configurer les semestres}
    \begin{itemize}
        \item S1 : Septembre 2024 - Janvier 2025
        \item S2 : Février 2025 - Juin 2025
    \end{itemize}

    \item \textbf{Préparer les types d'examens}
    \begin{itemize}
        \item CC : 30\%
        \item TP : 20\%
        \item Examen Final : 50\%
    \end{itemize}

    \item \textbf{Importer les nouveaux étudiants}
    \begin{itemize}
        \item Télécharger modèle Excel
        \item Remplir avec liste des nouveaux
        \item Importer
        \item Vérifier matricules générés
    \end{itemize}

    \item \textbf{Inscrire les étudiants}
    \begin{itemize}
        \item Inscrire dans leurs classes respectives
        \item Marquer statut : Nouveau / Redoublant
    \end{itemize}

    \item \textbf{Préparer la saisie des notes}
    \begin{itemize}
        \item Planifier les examens du S1
        \item Créer les délégations pour enseignants
    \end{itemize}
\end{enumerate}

\subsection{Scénario 2 : Session d'examens}

\subsubsection{Contexte}

Fin du semestre 1, vous devez organiser la session d'examens et la saisie des notes.

\subsubsection{Workflow complet}

\begin{enumerate}[leftmargin=*]
    \item \textbf{Planification} (Administrateur)
    \begin{itemize}
        \item Menu \menu{Planification des examens}
        \item Générer tous les examens du S1
        \item Ajuster les dates si nécessaire
        \item Définir dates limites de saisie
    \end{itemize}

    \item \textbf{Délégation} (Administrateur)
    \begin{itemize}
        \item Pour chaque cours :
        \item Créer délégation pour l'enseignant
        \item Période : date examen jusqu'à +1 semaine
    \end{itemize}

    \item \textbf{Notification automatique}
    \begin{itemize}
        \item Les enseignants reçoivent une notification
        \item Rappel de la date limite de saisie
    \end{itemize}

    \item \textbf{Saisie} (Enseignants)
    \begin{itemize}
        \item Menu \menu{Saisie des notes}
        \item Sélectionner cours et type d'examen
        \item Saisir les notes ou importer Excel
        \item \bouton{Enregistrer}
    \end{itemize}

    \item \textbf{Validation} (Enseignants)
    \begin{itemize}
        \item Relire les notes
        \item \bouton{Soumettre pour validation}
    \end{itemize}

    \item \textbf{Approbation niveau 1} (Dean)
    \begin{itemize}
        \item Consulter les notes soumises
        \item Vérifier cohérence
        \item \bouton{Approuver} ou \bouton{Rejeter}
    \end{itemize}

    \item \textbf{Approbation niveau 2} (Administrateur)
    \begin{itemize}
        \item Validation finale
        \item \bouton{Approuver}
    \end{itemize}

    \item \textbf{Publication}
    \begin{itemize}
        \item Les notes sont visibles aux étudiants
        \item Calcul automatique des moyennes
        \item Attribution automatique des crédits
    \end{itemize}
\end{enumerate}

\subsection{Scénario 3 : Correction d'une note}

\subsubsection{Contexte}

Un enseignant découvre une erreur dans une note déjà validée.

\subsubsection{Procédure de modification}

\begin{enumerate}[leftmargin=*]
    \item \textbf{Identification} (Enseignant)
    \begin{itemize}
        \item Étudiant : Dupont Jean
        \item Cours : Anatomie
        \item Note actuelle : 8/20
        \item Note correcte : 18/20 (erreur de saisie)
    \end{itemize}

    \item \textbf{Création du workflow} (Enseignant)
    \begin{itemize}
        \item Menu \menu{Workflows} > \bouton{Nouvelle demande}
        \item Type : Modification de note
        \item Sélectionner l'étudiant et le cours
        \item Ancienne valeur : 8
        \item Nouvelle valeur : 18
        \item Justification : "Erreur de saisie, confusion avec un autre étudiant"
        \item \bouton{Soumettre}
    \end{itemize}

    \item \textbf{Notification}
    \begin{itemize}
        \item Le Dean reçoit une notification
    \end{itemize}

    \item \textbf{Validation niveau 1} (Dean)
    \begin{itemize}
        \item Examine la demande
        \item Vérifie la justification
        \item Peut contacter l'enseignant si doute
        \item \bouton{Approuver niveau 1}
    \end{itemize}

    \item \textbf{Notification niveau 2}
    \begin{itemize}
        \item L'Administrateur reçoit notification
    \end{itemize}

    \item \textbf{Validation niveau 2} (Administrateur)
    \begin{itemize}
        \item Révision finale
        \item \bouton{Approuver niveau 2}
    \end{itemize}

    \item \textbf{Exécution automatique}
    \begin{itemize}
        \item La note passe automatiquement de 8 à 18
        \item La moyenne du cours est recalculée
        \item La moyenne générale est mise à jour
        \item Les crédits sont attribués si applicable
        \item L'étudiant reçoit une notification
    \end{itemize}

    \item \textbf{Traçabilité}
    \begin{itemize}
        \item Historique complet conservé
        \item Visible dans l'audit trail
    \end{itemize}
\end{enumerate}

\subsection{Scénario 4 : Génération de relevés de notes}

\subsubsection{Contexte}

Fin de semestre, vous devez générer et distribuer les relevés de notes à tous les étudiants d'une classe.

\subsubsection{Procédure}

\begin{enumerate}[leftmargin=*]
    \item \textbf{Préparation}
    \begin{itemize}
        \item Vérifier que toutes les notes sont validées
        \item Vérifier que les logos sont en place
        \item Tester avec 1-2 étudiants d'abord
    \end{itemize}

    \item \textbf{Configuration de l'export}
    \begin{itemize}
        \item Menu \menu{Exports} > \menu{Relevés de notes}
        \item Sélectionner classe : Médecine Année 1
        \item Semestre : S1
        \item Options :
        \begin{itemize}
            \item ☑ Inclure logo
            \item ☑ Afficher crédits ECTS
            \item ☑ Afficher mentions
        \end{itemize}
    \end{itemize}

    \item \textbf{Génération}
    \begin{itemize}
        \item Format : ZIP (PDFs individuels)
        \item \bouton{Générer}
        \item Attendre la fin (barre de progression)
    \end{itemize}

    \item \textbf{Téléchargement}
    \begin{itemize}
        \item Télécharger l'archive ZIP
        \item Extraire localement
    \end{itemize}

    \item \textbf{Vérification}
    \begin{itemize}
        \item Ouvrir quelques PDFs au hasard
        \item Vérifier :
        \begin{itemize}
            \item Logos présents
            \item Informations correctes
            \item Calculs de moyennes
            \item Mise en page propre
        \end{itemize}
    \end{itemize}

    \item \textbf{Distribution}
    \begin{itemize}
        \item Envoyer par email aux étudiants
        \item Ou imprimer et distribuer physiquement
        \item Archiver une copie pour traçabilité
    \end{itemize}
\end{enumerate}

\subsection{Scénario 5 : Gestion d'un redoublant}

\subsubsection{Contexte}

Un étudiant a échoué au S1 (moyenne $<$ 10). Il doit redoubler certains cours au S2.

\subsubsection{Gestion}

\begin{enumerate}[leftmargin=*]
    \item \textbf{Identification}
    \begin{itemize}
        \item Étudiant : Martin Paul
        \item Moyenne S1 : 8.5/20
        \item Cours échoués : Anatomie (7/20), Chimie (6/20)
        \item Cours validés : Physiologie (12/20), Biologie (11/20)
    \end{itemize}

    \item \textbf{Inscription sélective S2}
    \begin{itemize}
        \item Ne pas inscrire aux cours déjà validés
        \item Inscrire uniquement :
        \begin{itemize}
            \item Anatomie (tentative 2)
            \item Chimie (tentative 2)
            \item Nouveaux cours du S2
        \end{itemize}
    \end{itemize}

    \item \textbf{Suivi}
    \begin{itemize}
        \item Le système trace le nombre de tentatives
        \item Visible dans le dossier de l'étudiant
    \end{itemize}

    \item \textbf{Validation finale}
    \begin{itemize}
        \item Si réussite aux cours redoublés :
        \begin{itemize}
            \item Remplacement des notes précédentes
            \item Attribution des crédits
            \item Calcul de la moyenne annuelle
        \end{itemize}
    \end{itemize}
\end{enumerate}

\subsection{Scénario 6 : Import massif d'étudiants}

\subsubsection{Contexte}

Début d'année, inscription de 200 nouveaux étudiants répartis sur 5 classes.

\subsubsection{Procédure optimisée}

\begin{enumerate}[leftmargin=*]
    \item \textbf{Préparation du fichier Excel}
    \begin{itemize}
        \item Télécharger le modèle
        \item Remplir toutes les colonnes :
        \begin{itemize}
            \item Nom, prénom
            \item Date de naissance
            \item Lieu de naissance
            \item Genre
            \item Email (optionnel)
            \item Classe cible
        \end{itemize}
        \item Sauvegarder en .xlsx
    \end{itemize}

    \item \textbf{Test avec échantillon}
    \begin{itemize}
        \item Créer fichier test avec 5 étudiants
        \item Importer
        \item Vérifier résultats
        \item Corriger erreurs éventuelles
    \end{itemize}

    \item \textbf{Import complet}
    \begin{itemize}
        \item Menu \menu{Étudiants} > \bouton{Importer}
        \item Sélectionner fichier complet
        \item Rapport de validation :
        \begin{itemize}
            \item 200 étudiants valides
            \item 0 erreur
        \end{itemize}
        \item \bouton{Confirmer l'import}
    \end{itemize}

    \item \textbf{Vérification post-import}
    \begin{itemize}
        \item Vérifier matricules générés
        \item Vérifier répartition par classe
        \item Exporter liste pour validation
    \end{itemize}

    \item \textbf{Inscription aux cours}
    \begin{itemize}
        \item Pour chaque classe :
        \item Inscription automatique au S1
        \item Vérifier nombre d'inscrits par cours
    \end{itemize}
\end{enumerate}

\subsection{Bonnes pratiques générales}

\begin{tcolorbox}[colback=green!5!white,colframe=green!75!black,title=Recommandations]

\textbf{Organisation :}
\begin{itemize}[leftmargin=*]
    \item Planifiez les tâches en début d'année académique
    \item Documentez vos procédures locales
    \item Formez les utilisateurs avant utilisation
    \item Préparez des modèles et templates
\end{itemize}

\textbf{Qualité des données :}
\begin{itemize}[leftmargin=*]
    \item Vérifiez toujours avant validation
    \item Testez avec échantillons
    \item Conservez les fichiers sources
    \item Maintenez la cohérence des données
\end{itemize}

\textbf{Communication :}
\begin{itemize}[leftmargin=*]
    \item Informez les utilisateurs des procédures
    \item Communiquez les délais
    \item Notifiez les changements importants
    \item Assurez un support réactif
\end{itemize}
\end{tcolorbox}

\newpage
