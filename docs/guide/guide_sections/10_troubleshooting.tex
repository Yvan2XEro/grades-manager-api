% Section 10 - Dépannage
\section{Dépannage et résolution de problèmes}

\subsection{Problèmes de connexion}

\subsubsection{Impossible de se connecter}

\begin{tcolorbox}[colback=red!5!white,colframe=red!75!black,title=Symptôme]
Message d'erreur : "Identifiants incorrects" ou "Échec de connexion"
\end{tcolorbox}

\textbf{Solutions :}
\begin{enumerate}[leftmargin=*]
    \item Vérifiez que l'email est correct (pas de majuscules par erreur)
    \item Vérifiez votre mot de passe (attention à Caps Lock)
    \item Utilisez "Mot de passe oublié ?" pour réinitialiser
    \item Vérifiez que le compte est actif (contactez l'administrateur)
    \item Videz le cache du navigateur
    \item Essayez en navigation privée
\end{enumerate}

\subsubsection{Session expirée}

\begin{tcolorbox}[colback=red!5!white,colframe=red!75!black,title=Symptôme]
Déconnexion automatique fréquente
\end{tcolorbox}

\textbf{Solutions :}
\begin{enumerate}[leftmargin=*]
    \item Cochez "Se souvenir de moi" à la connexion
    \item Vérifiez que les cookies sont activés
    \item Augmentez la durée de session (administrateurs uniquement)
    \item Vérifiez la synchronisation de l'horloge système
\end{enumerate}

\subsection{Problèmes de saisie de notes}

\subsubsection{Impossible de saisir les notes}

\begin{tcolorbox}[colback=red!5!white,colframe=red!75!black,title=Symptôme]
Champs de saisie désactivés ou message "Non autorisé"
\end{tcolorbox}

\textbf{Causes possibles :}
\begin{enumerate}[leftmargin=*]
    \item Vous n'avez pas la permission \code{canGrade}
    \item L'examen est déjà clôturé (date limite dépassée)
    \item Vous n'avez pas de délégation pour cet examen
    \item Les notes ont déjà été validées (modification nécessite workflow)
\end{enumerate}

\textbf{Solutions :}
\begin{enumerate}[leftmargin=*]
    \item Vérifiez votre rôle (Teacher minimum requis)
    \item Vérifiez la date limite de saisie de l'examen
    \item Demandez une délégation à l'administrateur
    \item Si notes validées : créez un workflow de modification
\end{enumerate}

\subsubsection{Notes rejetées à l'import Excel}

\begin{tcolorbox}[colback=red!5!white,colframe=red!75!black,title=Symptôme]
Message d'erreur lors de l'import de notes
\end{tcolorbox}

\textbf{Vérifications :}
\begin{enumerate}[leftmargin=*]
    \item Format du fichier : doit être .xlsx ou .xls
    \item Structure des colonnes : utilisez le modèle fourni
    \item Format des notes :
    \begin{itemize}
        \item Nombres entre 0 et 20
        \item Séparateur décimal : point (15.5) ou virgule (15,5)
        \item Pas de texte dans les cellules de notes
    \end{itemize}
    \item Matricules : doivent correspondre exactement à la base
\end{enumerate}

\subsection{Problèmes de calcul}

\subsubsection{Moyenne incorrecte}

\begin{tcolorbox}[colback=red!5!white,colframe=red!75!black,title=Symptôme]
La moyenne affichée ne correspond pas aux notes
\end{tcolorbox}

\textbf{Vérifications :}
\begin{enumerate}[leftmargin=*]
    \item Vérifiez les poids des types d'examens (doivent totaliser 100\%)
    \item Vérifiez les coefficients des cours dans l'UE
    \item Assurez-vous que toutes les notes sont saisies
    \item Vérifiez qu'il n'y a pas de notes en attente de validation
    \item Rafraîchissez la page (Ctrl+F5)
\end{enumerate}

Si le problème persiste :
\begin{lstlisting}[language={}, frame=single]
# Recalculer toutes les moyennes (administrateur)
bun run --filter server recalculate-grades
\end{lstlisting}

\subsection{Problèmes d'export}

\subsubsection{Export PDF vide ou incomplet}

\begin{tcolorbox}[colback=red!5!white,colframe=red!75!black,title=Symptôme]
Le PDF généré est vide ou manque des informations
\end{tcolorbox}

\textbf{Solutions :}
\begin{enumerate}[leftmargin=*]
    \item Vérifiez que les notes sont validées
    \item Vérifiez le template HTML (erreurs de syntaxe Handlebars)
    \item Vérifiez que les logos existent et sont accessibles
    \item Regardez les logs serveur pour les erreurs
    \item Testez avec un seul étudiant d'abord
\end{enumerate}

\subsubsection{Logos manquants}

\begin{tcolorbox}[colback=red!5!white,colframe=red!75!black,title=Symptôme]
Les logos n'apparaissent pas dans les exports
\end{tcolorbox}

\textbf{Vérifications :}
\begin{enumerate}[leftmargin=*]
    \item Fichiers présents dans \code{apps/server/storage/logos/}
    \item Noms corrects : \code{logo-fmsp.png}, \code{logo-ud.png}
    \item Format PNG (pas JPG ou autre)
    \item Permissions de lecture sur les fichiers
    \item Redémarrez le serveur après ajout de logos
\end{enumerate}

\subsection{Problèmes de performance}

\subsubsection{Application lente}

\begin{tcolorbox}[colback=red!5!white,colframe=red!75!black,title=Symptôme]
Temps de chargement élevés, interface qui rame
\end{tcolorbox}

\textbf{Diagnostics :}
\begin{enumerate}[leftmargin=*]
    \item \textbf{Côté client :}
    \begin{itemize}
        \item Ouvrez la console développeur (F12)
        \item Onglet Network : vérifiez les temps de requêtes
        \item Vérifiez la vitesse de votre connexion Internet
        \item Essayez un autre navigateur
    \end{itemize}

    \item \textbf{Côté serveur :}
    \begin{itemize}
        \item Vérifiez la charge CPU : \code{top} ou \code{htop}
        \item Vérifiez la mémoire disponible : \code{free -h}
        \item Vérifiez l'espace disque : \code{df -h}
        \item Consultez les logs pour erreurs
    \end{itemize}
\end{enumerate}

\textbf{Optimisations :}
\begin{enumerate}[leftmargin=*]
    \item Optimiser la base de données :
    \begin{lstlisting}[language={}, frame=single]
-- Analyser et optimiser
VACUUM ANALYZE;

-- Reconstruire les index
REINDEX DATABASE grades_manager;
    \end{lstlisting}

    \item Nettoyer le cache :
    \begin{lstlisting}[language={}, frame=single]
# Vider le cache de l'application
rm -rf apps/server/cache/*
    \end{lstlisting}

    \item Augmenter les ressources serveur si nécessaire
\end{enumerate}

\subsection{Problèmes de base de données}

\subsubsection{Erreur de connexion à la base}

\begin{tcolorbox}[colback=red!5!white,colframe=red!75!black,title=Symptôme]
\code{Error: Connection refused} ou \code{ECONNREFUSED}
\end{tcolorbox}

\textbf{Solutions :}
\begin{enumerate}[leftmargin=*]
    \item Vérifiez que PostgreSQL est démarré :
    \begin{lstlisting}[language={}, frame=single]
# Linux
sudo systemctl status postgresql

# Si arrete, demarrer
sudo systemctl start postgresql
    \end{lstlisting}

    \item Vérifiez le \code{DATABASE\_URL} dans \code{.env}
    \item Testez la connexion manuellement :
    \begin{lstlisting}[language={}, frame=single]
psql -U grades_app -d grades_manager
    \end{lstlisting}

    \item Vérifiez les permissions PostgreSQL dans \code{pg\_hba.conf}
\end{enumerate}

\subsubsection{Migration échouée}

\begin{tcolorbox}[colback=red!5!white,colframe=red!75!black,title=Symptôme]
Erreur lors de \code{bun db:push} ou \code{bun db:migrate}
\end{tcolorbox}

\textbf{Solutions :}
\begin{enumerate}[leftmargin=*]
    \item Vérifiez les logs d'erreur
    \item Vérifiez que la base de données est accessible
    \item Si conflit de schéma, restaurez depuis sauvegarde :
    \begin{lstlisting}[language={}, frame=single]
# Sauvegarder d'abord
pg_dump grades_manager > backup_before_fix.sql

# Restaurer depuis une sauvegarde propre
psql grades_manager < backup_clean.sql

# Relancer la migration
bun db:migrate
    \end{lstlisting}
\end{enumerate}

\subsection{Messages d'erreur courants}

\begin{table}[h]
\centering
\footnotesize
\begin{tabular}{|p{5cm}|p{8cm}|}
\hline
\textbf{Message} & \textbf{Solution} \\
\hline
FORBIDDEN & Permissions insuffisantes, vérifiez votre rôle \\
\hline
UNAUTHORIZED & Non connecté, reconnectez-vous \\
\hline
NOT\_FOUND & Ressource inexistante, vérifiez l'ID \\
\hline
VALIDATION\_ERROR & Données invalides, vérifiez le format \\
\hline
DUPLICATE\_KEY & Doublon, l'élément existe déjà \\
\hline
FOREIGN\_KEY\_VIOLATION & Référence manquante, créez d'abord l'élément parent \\
\hline
\end{tabular}
\caption{Messages d'erreur fréquents}
\end{table}

\subsection{Obtenir de l'aide}

\subsubsection{Support technique}

\begin{itemize}[leftmargin=*]
    \item \textbf{Email} : cedrictefoye@gmail.com
    \item \textbf{Documentation} : Consultez ce guide et CLAUDE.md
    \item \textbf{Logs} : Conservez les logs d'erreur pour le support
\end{itemize}

\subsubsection{Informations à fournir}

Lors d'une demande de support, incluez :
\begin{enumerate}[leftmargin=*]
    \item Description du problème
    \item Étapes pour reproduire
    \item Messages d'erreur exacts
    \item Captures d'écran
    \item Version de l'application
    \item Navigateur et système d'exploitation
    \item Logs pertinents
\end{enumerate}

\newpage
