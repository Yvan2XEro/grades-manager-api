% Section 06 - Gestion des utilisateurs et autorisations
\section{Gestion des utilisateurs et autorisations}

\subsection{Système d'authentification}

Le système utilise \textbf{Better-Auth} avec séparation entre comptes d'authentification et profils métier.

\subsubsection{Architecture}

\begin{itemize}[leftmargin=*]
    \item \textbf{User (Better-Auth)} : Compte de connexion (email/password)
    \item \textbf{Domain User} : Profil métier avec rôle et permissions
    \item Lien entre les deux via \code{authUserId}
\end{itemize}

\subsection{Création d'utilisateurs}

\subsubsection{Ajout d'un utilisateur}

\begin{enumerate}[leftmargin=*]
    \item Menu \menu{Utilisateurs} > \bouton{Nouvel utilisateur}
    \item Remplissez le formulaire d'authentification :
    \begin{itemize}
        \item \champ{Email} : Adresse email unique
        \item \champ{Nom complet}
        \item \champ{Mot de passe} : Minimum 8 caractères
    \end{itemize}
    \item Le compte Better-Auth est créé
    \item Ensuite, créez le profil de domaine :
    \begin{itemize}
        \item Menu \menu{Profils de domaine} > \bouton{Nouveau profil}
        \item Liez au compte d'authentification
        \item Sélectionnez le \champ{Rôle}
        \item Remplissez les informations métier
    \end{itemize}
\end{enumerate}

\subsection{Rôles et hiérarchie}

\subsubsection{Rôles disponibles}

\begin{table}[h]
\centering
\begin{tabular}{|l|c|p{7cm}|}
\hline
\textbf{Rôle} & \textbf{Niveau} & \textbf{Description} \\
\hline
Super Admin & 6 & Contrôle total du système \\
Administrator & 5 & Gestion académique et administrative \\
Dean & 4 & Validation et supervision \\
Teacher & 3 & Saisie des notes et consultations \\
Staff & 2 & Personnel administratif \\
Student & 1 & Consultation des résultats personnels \\
\hline
\end{tabular}
\caption{Hiérarchie des rôles}
\end{table}

\subsubsection{Principe hiérarchique}

Les rôles sont hiérarchiques :
\begin{itemize}[leftmargin=*]
    \item Un rôle supérieur inclut toutes les permissions des rôles inférieurs
    \item \code{roleSatisfies(userRole, requiredRole)} vérifie si l'utilisateur a les permissions suffisantes
    \item Exemple : un Administrator peut faire tout ce qu'un Teacher peut faire
\end{itemize}

\subsection{Permissions}

\subsubsection{Permissions calculées}

Le système calcule automatiquement les permissions selon le rôle :

\begin{lstlisting}[frame=single, basicstyle=\small\ttfamily]
{
  canManageCatalog: true,      // Gerer programmes, cours
  canGrade: true,              // Saisir les notes
  canManageUsers: true,        // Gerer les utilisateurs
  canApproveLevel1: true,      // Valider workflows niveau 1
  canApproveLevel2: true,      // Valider workflows niveau 2
  canViewAllGrades: true,      // Voir toutes les notes
  canExport: true,             // Generer des exports
  canActivateAcademicYear: false // Activer annees (super_admin uniquement)
}
\end{lstlisting}

\subsubsection{Matrice de permissions}

\begin{table}[h]
\centering
\footnotesize
\begin{tabular}{|l|c|c|c|c|c|c|}
\hline
\textbf{Permission} & \textbf{Student} & \textbf{Staff} & \textbf{Teacher} & \textbf{Dean} & \textbf{Admin} & \textbf{S.Admin} \\
\hline
canManageCatalog & ✗ & ✗ & ✗ & ✗ & ✓ & ✓ \\
canGrade & ✗ & ✗ & ✓ & ✓ & ✓ & ✓ \\
canManageUsers & ✗ & ✗ & ✗ & ✗ & ✓ & ✓ \\
canApproveLevel1 & ✗ & ✗ & ✗ & ✓ & ✓ & ✓ \\
canApproveLevel2 & ✗ & ✗ & ✗ & ✗ & ✓ & ✓ \\
canViewAllGrades & ✗ & ✗ & ✗ & ✓ & ✓ & ✓ \\
canExport & ✗ & ✗ & ✗ & ✓ & ✓ & ✓ \\
canActivateAcademicYear & ✗ & ✗ & ✗ & ✗ & ✗ & ✓ \\
\hline
\end{tabular}
\caption{Matrice complète des permissions}
\end{table}

\subsection{Gestion des profils de domaine}

\subsubsection{Informations du profil}

Un profil de domaine contient :
\begin{itemize}[leftmargin=*]
    \item Nom, prénom, email
    \item Rôle dans le système
    \item Faculté de rattachement (optionnel)
    \item Département ou service
    \item Numéro de téléphone
    \item Photo de profil (optionnel)
\end{itemize}

\subsubsection{Modification d'un profil}

\begin{enumerate}[leftmargin=*]
    \item Menu \menu{Profils de domaine}
    \item Sélectionnez le profil à modifier
    \item Modifiez les informations
    \item \bouton{Enregistrer}
\end{enumerate}

\attention{La modification du rôle nécessite des permissions Administrator ou supérieures.}

\subsection{Gestion des mots de passe}

\subsubsection{Réinitialisation par l'utilisateur}

Si l'utilisateur a oublié son mot de passe :
\begin{enumerate}[leftmargin=*]
    \item Sur la page de connexion, cliquer sur "Mot de passe oublié ?"
    \item Entrer son adresse email
    \item Suivre les instructions de l'email reçu
    \item Créer un nouveau mot de passe
\end{enumerate}

\subsubsection{Réinitialisation par l'administrateur}

Un administrateur peut réinitialiser le mot de passe d'un utilisateur :
\begin{enumerate}[leftmargin=*]
    \item Menu \menu{Utilisateurs}
    \item Sélectionnez l'utilisateur
    \item Cliquez sur \bouton{Réinitialiser le mot de passe}
    \item Un nouveau mot de passe temporaire est généré
    \item Communiquez-le à l'utilisateur de manière sécurisée
    \item L'utilisateur devra le changer à la première connexion
\end{enumerate}

\subsection{Sessions et sécurité}

\subsubsection{Durée des sessions}

\begin{itemize}[leftmargin=*]
    \item Session par défaut : 7 jours
    \item Option "Se souvenir de moi" : 30 jours
    \item Déconnexion automatique après inactivité (configurable)
\end{itemize}

\subsubsection{Bonnes pratiques de sécurité}

\begin{tcolorbox}[colback=orange!5!white,colframe=orange!75!black,title=Sécurité des comptes]

\textbf{Mots de passe :}
\begin{itemize}[leftmargin=*]
    \item Minimum 8 caractères
    \item Mélange de majuscules, minuscules, chiffres et symboles
    \item Ne jamais réutiliser le même mot de passe
    \item Changer régulièrement (tous les 3-6 mois)
\end{itemize}

\textbf{Utilisation :}
\begin{itemize}[leftmargin=*]
    \item Ne jamais partager ses identifiants
    \item Se déconnecter après chaque session
    \item Ne pas laisser une session ouverte sans surveillance
    \item Signaler immédiatement toute activité suspecte
\end{itemize}

\textbf{Administrateurs :}
\begin{itemize}[leftmargin=*]
    \item Limiter le nombre de Super Admins (2-3 maximum)
    \item Réviser régulièrement les accès
    \item Désactiver les comptes inutilisés
    \item Auditer les logs d'authentification
\end{itemize}
\end{tcolorbox}

\subsection{Audit et traçabilité}

\subsubsection{Logs d'authentification}

Le système trace :
\begin{itemize}[leftmargin=*]
    \item Connexions réussies et échouées
    \item Adresses IP
    \item Horodatages
    \item Navigateur et système d'exploitation
\end{itemize}

\subsubsection{Actions utilisateurs}

Toutes les actions importantes sont tracées :
\begin{itemize}[leftmargin=*]
    \item Création/modification de données
    \item Saisie et modification de notes
    \item Approbations de workflows
    \item Exports de documents
\end{itemize}

\newpage
