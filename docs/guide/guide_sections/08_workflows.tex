% Section 08 - Système de workflows
\section{Système de workflows et notifications}

\subsection{Workflows d'approbation}

Le système de workflows permet de gérer les demandes de modification de notes avec validation hiérarchique.

\subsubsection{Principe}

\begin{itemize}[leftmargin=*]
    \item \textbf{Demandeur} : Enseignant détectant une erreur
    \item \textbf{Niveau 1} : Validation par le Dean
    \item \textbf{Niveau 2} : Validation finale par l'Administrateur
    \item \textbf{Exécution} : Application automatique si approuvé
\end{itemize}

\subsection{Création d'une demande}

\begin{enumerate}[leftmargin=*]
    \item Menu \menu{Workflows} > \bouton{Nouvelle demande}
    \item Remplissez :
    \begin{itemize}
        \item \champ{Type} : Modification de note
        \item \champ{Étudiant} concerné
        \item \champ{Cours} et \champ{Type d'examen}
        \item \champ{Valeur actuelle} : Note actuelle
        \item \champ{Nouvelle valeur} : Note corrigée
        \item \champ{Justification} : Raison de la modification
    \end{itemize}
    \item \bouton{Soumettre}
\end{enumerate}

\subsection{Traitement des demandes}

\subsubsection{Pour le Dean (Niveau 1)}

\begin{enumerate}[leftmargin=*]
    \item Accédez aux demandes en attente : Menu \menu{Workflows} > Filtre "En attente niveau 1"
    \item Consultez la demande :
    \begin{itemize}
        \item Informations de l'étudiant
        \item Note actuelle vs nouvelle note
        \item Justification de l'enseignant
    \end{itemize}
    \item Décision :
    \begin{itemize}
        \item \bouton{Approuver} : La demande passe au niveau 2
        \item \bouton{Rejeter} : Indiquez la raison du rejet
    \end{itemize}
\end{enumerate}

\subsubsection{Pour l'Administrateur (Niveau 2)}

\begin{enumerate}[leftmargin=*]
    \item Filtrer : "Approuvé niveau 1, en attente niveau 2"
    \item Révision finale
    \item Décision :
    \begin{itemize}
        \item \bouton{Approuver} : La note est modifiée automatiquement
        \item \bouton{Rejeter} : La demande est refusée définitivement
    \end{itemize}
\end{enumerate}

\subsection{États des workflows}

\begin{table}[h]
\centering
\begin{tabular}{|l|p{10cm}|}
\hline
\textbf{État} & \textbf{Description} \\
\hline
Pending L1 & En attente de validation niveau 1 (Dean) \\
Approved L1 & Approuvé niveau 1, en attente niveau 2 \\
Rejected L1 & Rejeté par le Dean \\
Approved L2 & Approuvé définitivement, exécuté \\
Rejected L2 & Rejeté par l'Administrateur \\
\hline
\end{tabular}
\caption{États possibles d'un workflow}
\end{table}

\subsection{Système de notifications}

\subsubsection{Types de notifications}

\begin{itemize}[leftmargin=*]
    \item \textbf{Workflow créé} : Notifie les validateurs
    \item \textbf{Workflow approuvé} : Notifie le demandeur et le niveau suivant
    \item \textbf{Workflow rejeté} : Notifie le demandeur avec raison
    \item \textbf{Note modifiée} : Notifie l'étudiant concerné
    \item \textbf{Examen planifié} : Rappel aux enseignants
    \item \textbf{Délégation expirée} : Alerte de fin de délégation
\end{itemize}

\subsubsection{Consultation des notifications}

\begin{enumerate}[leftmargin=*]
    \item Icône de notification (cloche) dans la barre de navigation
    \item Badge indiquant le nombre de notifications non lues
    \item Cliquez pour voir la liste
    \item Cliquez sur une notification pour accéder au détail
\end{enumerate}

\subsubsection{Gestion des notifications}

\begin{itemize}[leftmargin=*]
    \item Marquer comme lue
    \item Archiver
    \item Paramétrer les préférences :
    \begin{itemize}
        \item Recevoir par email
        \item Fréquence des emails (immédiate, quotidienne, hebdomadaire)
        \item Types de notifications à recevoir
    \end{itemize}
\end{itemize}

\subsection{Traçabilité}

\subsubsection{Historique des workflows}

Toutes les actions sont tracées :
\begin{itemize}[leftmargin=*]
    \item Date et heure de création
    \item Auteur de la demande
    \item Approbateurs successifs
    \item Commentaires à chaque étape
    \item Résultat final
\end{itemize}

\subsubsection{Consultation de l'historique}

\begin{enumerate}[leftmargin=*]
    \item Menu \menu{Workflows} > \bouton{Historique}
    \item Filtres disponibles :
    \begin{itemize}
        \item Par état
        \item Par demandeur
        \item Par période
        \item Par cours/étudiant
    \end{itemize}
    \item Export possible en Excel ou PDF
\end{enumerate}

\newpage
