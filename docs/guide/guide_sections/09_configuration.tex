% Section 09 - Configuration et administration
\section{Configuration et administration système}

\subsection{Paramètres généraux}

\subsubsection{Configuration de l'établissement}

Menu \menu{Configuration} > \menu{Établissement}

\begin{itemize}[leftmargin=*]
    \item \champ{Nom officiel} : Nom complet de l'établissement
    \item \champ{Nom court} : Acronyme (ex: FMSP)
    \item \champ{Adresse} : Adresse postale complète
    \item \champ{Téléphone} : Numéro principal
    \item \champ{Email} : Email de contact
    \item \champ{Site web} : URL du site officiel
    \item \champ{Logo principal} : Upload du logo
\end{itemize}

\subsubsection{Paramètres académiques}

\begin{itemize}[leftmargin=*]
    \item \textbf{Seuil de validation} : Note minimale pour valider un cours (défaut: 10/20)
    \item \textbf{Système de notation} : Sur 20, sur 100, ou lettres (A-F)
    \item \textbf{Mentions} :
    \begin{itemize}
        \item Très Bien : $\geq$ 16/20
        \item Bien : 14-15.99/20
        \item Assez Bien : 12-13.99/20
        \item Passable : 10-11.99/20
        \item Ajourné : $<$ 10/20
    \end{itemize}
    \item \textbf{Règles de progression} :
    \begin{itemize}
        \item Crédits requis pour passer au niveau suivant
        \item Nombre maximum de redoublements
        \item Conditions de compensation
    \end{itemize}
\end{itemize}

\subsection{Génération des numéros de matricule}

\subsubsection{Configuration du format}

Menu \menu{Configuration} > \menu{Matricules}

Le système génère automatiquement les matricules selon un format configurable :

\begin{lstlisting}[frame=single]
Format: {année}{faculté}{programme}{numéro séquentiel}
Exemple: 24MED001, 24MED002, ...
\end{lstlisting}

Composants :
\begin{itemize}[leftmargin=*]
    \item \code{\{année\}} : Deux derniers chiffres de l'année (24 pour 2024)
    \item \code{\{faculté\}} : Code de la faculté (ex: F pour FMSP)
    \item \code{\{programme\}} : Code du programme (ex: MED)
    \item \code{\{séquentiel\}} : Numéro auto-incrémenté (001, 002...)
\end{itemize}

\subsection{Gestion des semestres}

\subsubsection{Configuration globale}

\begin{itemize}[leftmargin=*]
    \item \textbf{Nombre de semestres par an} : Généralement 2
    \item \textbf{Dates de début/fin} :
    \begin{itemize}
        \item S1 : Septembre - Janvier
        \item S2 : Février - Juin
    \end{itemize}
    \item \textbf{Périodes d'examens} :
    \begin{itemize}
        \item Session normale
        \item Session de rattrapage
    \end{itemize}
\end{itemize}

\subsection{Templates et personnalisation}

\subsubsection{Templates de documents}

Les templates sont situés dans :
\begin{lstlisting}[frame=single]
apps/server/src/modules/exports/templates/
\end{lstlisting}

Fichiers disponibles :
\begin{itemize}[leftmargin=*]
    \item \code{teaching-unit-publication.html} : Relevé de notes
    \item \code{evaluation-publication.html} : Attestations
    \item Utilisation de Handlebars pour les variables dynamiques
\end{itemize}

\subsubsection{Variables disponibles}

\begin{lstlisting}[frame=single, basicstyle=\small\ttfamily]
{
  studentName: "Nom de l'etudiant",
  studentId: "Matricule",
  className: "Classe",
  semester: "S1",
  academicYear: "2024-2025",
  courses: [
    {
      code: "ANAT101",
      title: "Anatomie",
      coefficient: 2,
      credits: 4,
      grades: { CC: 12, TP: 14, EF: 16 },
      average: 14.4
    }
  ],
  overallAverage: 13.5,
  mention: "Assez Bien",
  status: "Admis"
}
\end{lstlisting}

\subsection{Sauvegarde et restauration}

\subsubsection{Sauvegarde de la base de données}

\textbf{Manuelle (PostgreSQL) :}
\begin{lstlisting}[language={}, frame=single]
# Sauvegarde complete
pg_dump -U grades_app grades_manager > backup_$(date +%Y%m%d).sql

# Sauvegarde comprimee
pg_dump -U grades_app grades_manager | gzip > backup.sql.gz
\end{lstlisting}

\textbf{Automatisée (cron Linux) :}
\begin{lstlisting}[language={}, frame=single]
# Ajouter dans crontab (crontab -e)
0 2 * * * pg_dump -U grades_app grades_manager |
  gzip > /backups/grades_$(date +\%Y\%m\%d).sql.gz
\end{lstlisting}

\subsubsection{Restauration}

\begin{lstlisting}[language={}, frame=single]
# Depuis une sauvegarde SQL
psql -U grades_app grades_manager < backup.sql

# Depuis une sauvegarde comprimee
gunzip -c backup.sql.gz | psql -U grades_app grades_manager
\end{lstlisting}

\attention{Testez toujours vos sauvegardes en les restaurant dans un environnement de test.}

\subsection{Monitoring et logs}

\subsubsection{Logs applicatifs}

Les logs sont stockés dans :
\begin{lstlisting}[frame=single]
apps/server/logs/
  ├── application.log    # Logs generaux
  ├── error.log          # Erreurs uniquement
  └── access.log         # Logs d'acces HTTP
\end{lstlisting}

\subsubsection{Niveau de logs}

Configurez le niveau dans le \code{.env} :
\begin{lstlisting}[frame=single]
LOG_LEVEL=info  # debug, info, warn, error
\end{lstlisting}

\subsubsection{Surveillance}

Points à surveiller :
\begin{itemize}[leftmargin=*]
    \item Espace disque disponible
    \item Charge de la base de données
    \item Temps de réponse des API
    \item Taux d'erreurs
    \item Nombre de sessions actives
\end{itemize}

\subsection{Maintenance}

\subsubsection{Tâches régulières}

\begin{table}[h]
\centering
\begin{tabular}{|l|l|l|}
\hline
\textbf{Tâche} & \textbf{Fréquence} & \textbf{Action} \\
\hline
Sauvegarde BD & Quotidienne & Script automatique \\
Nettoyage logs & Hebdomadaire & Archiver logs > 7 jours \\
Vérification espace & Quotidienne & Alertes si $<$ 10\% libre \\
Mise à jour dépendances & Mensuelle & \code{bun update} \\
Analyse performances & Mensuelle & Optimiser requêtes lentes \\
Audit sécurité & Trimestrielle & Réviser accès et logs \\
\hline
\end{tabular}
\caption{Planning de maintenance recommandé}
\end{table}

\subsubsection{Maintenance planifiée}

Pour une maintenance sans interruption :
\begin{enumerate}[leftmargin=*]
    \item Prévenir les utilisateurs 48h à l'avance
    \item Planifier hors heures de bureau (nuit/weekend)
    \item Effectuer une sauvegarde complète
    \item Appliquer les mises à jour
    \item Tester les fonctionnalités critiques
    \item Vérifier les logs
    \item Confirmer la remise en service
\end{enumerate}

\newpage
