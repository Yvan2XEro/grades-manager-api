% Section 04 - Gestion des étudiants
\section{Gestion des étudiants et inscriptions}

\subsection{Enregistrement des étudiants}

\subsubsection{Ajout manuel d'un étudiant}

\begin{enumerate}[leftmargin=*]
    \item Menu \menu{Étudiants} > \bouton{Nouvel étudiant}
    \item Remplissez le formulaire :
    \begin{itemize}
        \item \champ{Nom} : Nom de famille
        \item \champ{Prénom(s)} : Prénoms complets
        \item \champ{Date de naissance} : JJ/MM/AAAA
        \item \champ{Lieu de naissance} : Ville, Pays
        \item \champ{Genre} : Masculin/Féminin
        \item \champ{Email} : Adresse email (optionnel)
        \item \champ{Téléphone} : Numéro de contact (optionnel)
    \end{itemize}
    \item \bouton{Créer}
\end{enumerate}

\info{Le matricule étudiant est généré automatiquement selon les règles configurées dans le module \code{registration-numbers}.}

\subsubsection{Import en masse depuis Excel}

Pour inscrire plusieurs étudiants simultanément :

\begin{enumerate}[leftmargin=*]
    \item Téléchargez le modèle Excel : \menu{Import/Export} > \bouton{Télécharger modèle étudiants}
    \item Remplissez le fichier avec les informations des étudiants
    \item Retournez dans \menu{Étudiants} > \bouton{Importer}
    \item Sélectionnez votre fichier Excel
    \item Vérifiez le rapport de validation :
    \begin{itemize}
        \item Nombre d'étudiants valides
        \item Erreurs éventuelles (doublons, données manquantes)
    \end{itemize}
    \item Cliquez sur \bouton{Confirmer l'import}
\end{enumerate}

\begin{figure}[H]
    \centering
    \fbox{\parbox{0.9\textwidth}{\centering\vspace{2cm}[Capture : Interface d'import Excel]\\\vspace{2cm}}}
    \caption{Interface d'import d'étudiants depuis Excel}
    \label{fig:student-import}
\end{figure}

\subsection{Inscriptions annuelles}

Chaque année académique, les étudiants doivent être inscrits dans une classe.

\subsubsection{Inscription individuelle}

\begin{enumerate}[leftmargin=*]
    \item Menu \menu{Inscriptions} > \bouton{Nouvelle inscription}
    \item Sélectionnez :
    \begin{itemize}
        \item \champ{Année académique} : Année active par défaut
        \item \champ{Étudiant} : Recherchez par nom ou matricule
        \item \champ{Classe} : Classe d'inscription
        \item \champ{Statut} : Nouveau / Redoublant / Transfert
    \end{itemize}
    \item \bouton{Inscrire}
\end{enumerate}

\subsubsection{Inscription en masse}

Pour inscrire toute une promotion :

\begin{enumerate}[leftmargin=*]
    \item Menu \menu{Inscriptions} > \bouton{Inscription en masse}
    \item Sélectionnez :
    \begin{itemize}
        \item La classe cible
        \item Importez la liste des étudiants (Excel)
    \end{itemize}
    \item Vérifiez le résumé
    \item \bouton{Confirmer}
\end{enumerate}

\subsection{Inscriptions aux cours}

Une fois inscrits dans une classe, les étudiants doivent être inscrits aux cours du semestre.

\subsubsection{Inscription automatique}

L'inscription aux cours peut se faire automatiquement :

\begin{enumerate}[leftmargin=*]
    \item Menu \menu{Inscriptions aux cours} > \bouton{Inscription automatique}
    \item Sélectionnez :
    \begin{itemize}
        \item \champ{Classe}
        \item \champ{Semestre} : S1 ou S2
    \end{itemize}
    \item Le système inscrit automatiquement tous les étudiants de la classe à tous les cours du semestre
    \item Vérifiez le résumé et confirmez
\end{enumerate}

\subsubsection{Inscription sélective}

Pour des cas particuliers (redoublements, dispenses) :

\begin{enumerate}[leftmargin=*]
    \item Menu \menu{Inscriptions aux cours} > \bouton{Inscription sélective}
    \item Sélectionnez l'étudiant
    \item Cochez les cours auxquels l'inscrire
    \item \bouton{Valider}
\end{enumerate}

\subsubsection{Gestion des tentatives}

Le système suit le nombre de tentatives pour chaque cours :

\begin{itemize}[leftmargin=*]
    \item \textbf{Tentative 1} : Première inscription au cours
    \item \textbf{Tentative 2+} : Redoublement du cours (échec précédent)
    \item Le nombre de tentatives est tracé automatiquement
    \item Peut servir pour des règles académiques (limite de tentatives)
\end{itemize}

\subsection{Système de crédits ECTS}

\subsubsection{Ledger de crédits}

Le \textit{Student Credit Ledger} est un système de comptabilité des crédits :

\begin{itemize}[leftmargin=*]
    \item Enregistre chaque attribution de crédits
    \item Trace la source (cours validé)
    \item Calcule le cumul par étudiant
    \item Permet de suivre la progression vers le diplôme
\end{itemize}

\subsubsection{Attribution automatique}

Les crédits sont attribués automatiquement lorsque :
\begin{itemize}[leftmargin=*]
    \item Un étudiant obtient la moyenne ($\geq$ 10/20) dans un cours
    \item Les notes sont validées par l'administration
    \item Le cours a des crédits ECTS configurés
\end{itemize}

\subsubsection{Consultation du solde de crédits}

\begin{enumerate}[leftmargin=*]
    \item Menu \menu{Ledger de crédits}
    \item Recherchez un étudiant
    \item Consultez :
    \begin{itemize}
        \item Total des crédits acquis
        \item Détail par cours et semestre
        \item Historique des attributions
    \end{itemize}
\end{enumerate}

\begin{figure}[H]
    \centering
    \fbox{\parbox{0.9\textwidth}{\centering\vspace{2cm}[Capture : Ledger de crédits d'un étudiant]\\\vspace{2cm}}}
    \caption{Visualisation du ledger de crédits}
    \label{fig:credit-ledger}
\end{figure}

\subsection{Recherche et filtres}

\subsubsection{Recherche d'étudiants}

L'interface offre plusieurs options de recherche :

\begin{itemize}[leftmargin=*]
    \item Par \textbf{nom} ou \textbf{prénom}
    \item Par \textbf{matricule}
    \item Par \textbf{classe}
    \item Par \textbf{statut} (actif, diplômé, suspendu)
\end{itemize}

\subsubsection{Filtres avancés}

\begin{itemize}[leftmargin=*]
    \item Filtrer par année académique
    \item Filtrer par programme ou cycle
    \item Filtrer par statut d'inscription
    \item Trier par nom, matricule, classe
\end{itemize}

\subsection{Modification et suppression}

\subsubsection{Modification d'un étudiant}

\begin{enumerate}[leftmargin=*]
    \item Dans la liste des étudiants, cliquez sur l'étudiant
    \item Modifiez les informations nécessaires
    \item \bouton{Enregistrer}
\end{enumerate}

\attention{La modification du matricule n'est généralement pas recommandée car il est utilisé comme référence dans tout le système.}

\subsubsection{Suppression d'un étudiant}

\begin{enumerate}[leftmargin=*]
    \item Sélectionnez l'étudiant
    \item Cliquez sur \bouton{Supprimer}
    \item Confirmez l'action
\end{enumerate}

\attention{La suppression n'est possible que si l'étudiant n'a aucune note enregistrée. Sinon, préférez marquer le statut comme "inactif".}

\subsection{Export des listes}

\subsubsection{Export Excel}

\begin{enumerate}[leftmargin=*]
    \item Menu \menu{Étudiants}
    \item Appliquez les filtres souhaités
    \item Cliquez sur \bouton{Exporter vers Excel}
    \item Le fichier Excel contient :
    \begin{itemize}
        \item Matricule, nom, prénom
        \item Date et lieu de naissance
        \item Classe actuelle
        \item Statut d'inscription
    \end{itemize}
\end{enumerate}

\subsubsection{Export PDF (liste de classe)}

Pour générer une liste de classe imprimable :

\begin{enumerate}[leftmargin=*]
    \item Filtrez par classe
    \item \bouton{Exporter en PDF}
    \item Le PDF inclut :
    \begin{itemize}
        \item En-tête de l'établissement
        \item Nom de la classe et année académique
        \item Liste des étudiants avec numéros
        \item Espace pour signatures
    \end{itemize}
\end{enumerate}

\subsection{Bonnes pratiques}

\begin{tcolorbox}[colback=green!5!white,colframe=green!75!black,title=Recommandations]

\textbf{Import en masse :}
\begin{itemize}[leftmargin=*]
    \item Vérifiez toujours le fichier Excel avant import
    \item Utilisez le modèle fourni pour éviter les erreurs
    \item Testez avec un petit lot avant un gros import
\end{itemize}

\textbf{Matricules :}
\begin{itemize}[leftmargin=*]
    \item Ne modifiez jamais un matricule une fois attribué
    \item Vérifiez les doublons avant import
    \item Documentez votre format de matricule
\end{itemize}

\textbf{Inscriptions :}
\begin{itemize}[leftmargin=*]
    \item Inscrivez les étudiants AVANT le début des cours
    \item Vérifiez les inscriptions aux cours pour chaque semestre
    \item Mettez à jour le statut (redoublant, transfert) correctement
\end{itemize}
\end{tcolorbox}

\newpage
