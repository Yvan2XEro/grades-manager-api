% Section 03 - Gestion de la structure académique
\section{Gestion de la structure académique}

Cette section explique comment configurer la hiérarchie académique de votre institution : années académiques, facultés, programmes, cycles d'études, classes, semestres, unités d'enseignement et cours.

\subsection{Hiérarchie académique}

La structure académique suit cette hiérarchie :

\begin{verbatim}
Année Académique (2024-2025)
└── Faculté (FMSP)
    └── Programme (Médecine)
        └── Cycle d'études (Licence)
            └── Classe (Médecine Année 1)
                └── Semestre (S1, S2)
                    └── Unité d'Enseignement (UE)
                        └── Cours (Anatomie, Physiologie, etc.)
\end{verbatim}

\subsection{Années académiques}

\subsubsection{Création d'une année académique}

\begin{enumerate}[leftmargin=*]
    \item Accédez au menu \menu{Années académiques}
    \item Cliquez sur \bouton{Nouvelle année}
    \item Remplissez les informations :
    \begin{itemize}
        \item \champ{Année de début} : Ex. 2024
        \item \champ{Année de fin} : Ex. 2025
        \item \champ{Date de début} : 01/09/2024
        \item \champ{Date de fin} : 31/08/2025
    \end{itemize}
    \item Cliquez sur \bouton{Créer}
\end{enumerate}

\subsubsection{Activation d'une année académique}

\attention{Seuls les Super Admins peuvent activer une année académique.}

\begin{enumerate}[leftmargin=*]
    \item Dans la liste des années, identifiez l'année à activer
    \item Cliquez sur \bouton{Activer}
    \item Confirmez l'action
\end{enumerate}

\info{Une seule année peut être active à la fois. L'activation d'une nouvelle année désactive automatiquement l'année précédente.}

\subsubsection{Conséquences de l'activation}

Lorsqu'une année académique est activée :
\begin{itemize}[leftmargin=*]
    \item Toutes les opérations (inscriptions, notes, examens) s'appliquent à cette année
    \item Les inscriptions de l'année précédente sont gelées
    \item Un nouveau cycle d'examens peut commencer
    \item Les étudiants peuvent être inscrits dans les nouvelles classes
\end{itemize}

\subsection{Facultés}

\subsubsection{Création d'une faculté}

\begin{enumerate}[leftmargin=*]
    \item Menu \menu{Facultés} > \bouton{Nouvelle faculté}
    \item Remplissez :
    \begin{itemize}
        \item \champ{Nom} : Ex. "Faculté de Médecine et Sciences Pharmaceutiques"
        \item \champ{Code} : Ex. "FMSP" (utilisé dans les exports)
        \item \champ{Description} : Texte descriptif (optionnel)
    \end{itemize}
    \item \bouton{Créer}
\end{enumerate}

\subsubsection{Gestion des facultés}

Depuis la liste des facultés, vous pouvez :
\begin{itemize}[leftmargin=*]
    \item Modifier les informations d'une faculté
    \item Consulter les programmes associés
    \item Supprimer une faculté (si aucun programme n'y est rattaché)
\end{itemize}

\subsection{Programmes}

Les programmes représentent les filières d'études (Médecine, Pharmacie, Soins Infirmiers, etc.).

\subsubsection{Création d'un programme}

\begin{enumerate}[leftmargin=*]
    \item Menu \menu{Programmes} > \bouton{Nouveau programme}
    \item Sélectionnez la \champ{Faculté} parente
    \item Remplissez :
    \begin{itemize}
        \item \champ{Nom} : Ex. "Médecine"
        \item \champ{Code} : Ex. "MED"
        \item \champ{Description} : Optionnel
        \item \champ{Durée} : Nombre d'années (ex: 7 pour Médecine)
    \end{itemize}
    \item \bouton{Créer}
\end{enumerate}

\subsection{Cycles d'études}

Les cycles représentent les niveaux académiques (Licence, Master, Doctorat).

\subsubsection{Création d'un cycle}

\begin{enumerate}[leftmargin=*]
    \item Menu \menu{Cycles d'études} > \bouton{Nouveau cycle}
    \item Sélectionnez le \champ{Programme}
    \item Remplissez :
    \begin{itemize}
        \item \champ{Nom} : Ex. "Licence"
        \item \champ{Code} : Ex. "L"
        \item \champ{Niveau} : 1, 2, ou 3 (L=1, M=2, D=3)
        \item \champ{Durée} : Nombre d'années (ex: 3 pour Licence)
    \end{itemize}
    \item \bouton{Créer}
\end{enumerate}

\subsection{Classes}

Les classes regroupent les étudiants d'une même année et promotion.

\subsubsection{Création d'une classe}

\begin{enumerate}[leftmargin=*]
    \item Menu \menu{Classes} > \bouton{Nouvelle classe}
    \item Sélectionnez le \champ{Cycle d'études}
    \item Remplissez :
    \begin{itemize}
        \item \champ{Nom} : Ex. "Médecine Année 1" ou "MED1"
        \item \champ{Code} : Ex. "MED1-2024"
        \item \champ{Année du cycle} : Position dans le cycle (1, 2, 3...)
        \item \champ{Capacité} : Nombre maximum d'étudiants (optionnel)
    \end{itemize}
    \item \bouton{Créer}
\end{enumerate}

\subsection{Semestres}

Les semestres divisent l'année académique en périodes.

\subsubsection{Configuration des semestres}

Généralement, deux semestres par année :

\begin{table}[h]
\centering
\begin{tabular}{|l|l|l|}
\hline
\textbf{Semestre} & \textbf{Code} & \textbf{Période} \\
\hline
Semestre 1 & S1 & Septembre - Janvier \\
Semestre 2 & S2 & Février - Juin \\
\hline
\end{tabular}
\caption{Semestres standards}
\end{table}

Les semestres sont généralement créés automatiquement, mais peuvent être personnalisés via le menu \menu{Semestres}.

\subsection{Unités d'Enseignement (UE)}

Les UE regroupent des cours par thématique.

\subsubsection{Création d'une UE}

\begin{enumerate}[leftmargin=*]
    \item Menu \menu{Unités d'enseignement} > \bouton{Nouvelle UE}
    \item Remplissez :
    \begin{itemize}
        \item \champ{Code} : Ex. "UE1-S1"
        \item \champ{Titre} : Ex. "Sciences Fondamentales"
        \item \champ{Description} : Optionnel
        \item \champ{Crédits ECTS} : Total de l'UE
    \end{itemize}
    \item \bouton{Créer}
\end{enumerate}

\subsection{Cours (Éléments Constitutifs)}

Les cours sont les matières individuelles enseignées aux étudiants.

\subsubsection{Création d'un cours}

\begin{enumerate}[leftmargin=*]
    \item Menu \menu{Cours} > \bouton{Nouveau cours}
    \item Sélectionnez l'\champ{Unité d'Enseignement}
    \item Remplissez :
    \begin{itemize}
        \item \champ{Code} : Ex. "ANAT101"
        \item \champ{Titre} : Ex. "Anatomie Générale"
        \item \champ{Semestre} : S1 ou S2
        \item \champ{Coefficient} : Poids dans le calcul de moyenne
        \item \champ{Crédits ECTS} : Crédits attribués si validé
        \item \champ{Enseignant responsable} : Optionnel
    \end{itemize}
    \item \bouton{Créer}
\end{enumerate}

\subsubsection{Import en masse des cours}

Pour gagner du temps, vous pouvez importer plusieurs cours depuis un fichier Excel :

\begin{enumerate}[leftmargin=*]
    \item Téléchargez le modèle Excel depuis \menu{Import/Export}
    \item Remplissez le fichier avec vos cours
    \item Importez via \bouton{Importer des cours}
    \item Vérifiez le rapport d'import
    \item Confirmez l'import
\end{enumerate}

\subsection{Bonnes pratiques}

\begin{tcolorbox}[colback=green!5!white,colframe=green!75!black,title=Conseils d'organisation]

\textbf{Codes cohérents :}
\begin{itemize}[leftmargin=*]
    \item Utilisez des codes courts et mnémoniques (MED, PHAR, L, M, D)
    \item Maintenez une convention de nommage cohérente
    \item Documentez votre système de codes
\end{itemize}

\textbf{Ordre de création :}
\begin{enumerate}[leftmargin=*]
    \item Créez d'abord l'année académique et activez-la
    \item Ensuite : Facultés → Programmes → Cycles → Classes
    \item Puis : UE → Cours
    \item Enfin : Étudiants et Inscriptions
\end{enumerate}

\textbf{Planification :}
\begin{itemize}[leftmargin=*]
    \item Préparez votre structure académique complète avant de commencer
    \item Documentez les coefficients et crédits de chaque cours
    \item Validez avec les responsables académiques avant la création
\end{itemize}
\end{tcolorbox}

\newpage
