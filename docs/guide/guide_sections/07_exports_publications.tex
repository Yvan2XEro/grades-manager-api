% Section 07 - Exports et publications
\section{Exports et publications}

\subsection{Relevés de notes}

\subsubsection{Génération de relevés}

\begin{enumerate}[leftmargin=*]
    \item Menu \menu{Exports} > \menu{Relevés de notes}
    \item Sélectionnez les critères :
    \begin{itemize}
        \item \champ{Classe}
        \item \champ{Semestre} : S1, S2 ou Annuel
        \item \champ{Année académique}
        \item \champ{Format} : PDF
    \end{itemize}
    \item Options :
    \begin{itemize}
        \item Inclure le logo de l'établissement
        \item Inclure les signatures
        \item Afficher les crédits ECTS
        \item Afficher les mentions
    \end{itemize}
    \item \bouton{Générer}
\end{enumerate}

\subsubsection{Contenu du relevé}

Un relevé de notes standard contient :
\begin{itemize}[leftmargin=*]
    \item En-tête : Logo, nom de l'établissement
    \item Informations étudiant : Matricule, nom, classe
    \item Tableau des notes :
    \begin{itemize}
        \item Liste des UE et cours
        \item Notes par type d'examen (CC, TP, EF)
        \item Moyenne du cours
        \item Coefficient et crédits
    \end{itemize}
    \item Récapitulatif :
    \begin{itemize}
        \item Moyenne générale
        \item Total des crédits acquis
        \item Mention obtenue
        \item Décision : Admis / Ajourné
    \end{itemize}
    \item Pied de page : Date, signatures
\end{itemize}

\begin{figure}[H]
    \centering
    \fbox{\parbox{0.9\textwidth}{\centering\vspace{3cm}[Capture : Exemple de relevé de notes]\\\vspace{3cm}}}
    \caption{Exemple de relevé de notes généré}
    \label{fig:releve-notes}
\end{figure}

\subsubsection{Personnalisation du template}

Les templates HTML sont situés dans :
\begin{lstlisting}[frame=single]
apps/server/src/modules/exports/templates/
\end{lstlisting}

Pour personnaliser :
\begin{enumerate}[leftmargin=*]
    \item Éditez le fichier \code{teaching-unit-publication.html}
    \item Utilisez la syntaxe Handlebars pour les variables :
    \begin{lstlisting}[language={}, frame=single, basicstyle=\small\ttfamily]
<h1>{{studentName}}</h1>
<p>Matricule: {{studentId}}</p>
{{#each courses}}
  <tr>
    <td>{{this.title}}</td>
    <td>{{this.average}}</td>
  </tr>
{{/each}}
    \end{lstlisting}
    \item Testez en générant un relevé
\end{enumerate}

\subsection{Attestations de réussite}

\subsubsection{Génération automatique}

\begin{enumerate}[leftmargin=*]
    \item Menu \menu{Exports} > \menu{Attestations}
    \item Sélectionnez :
    \begin{itemize}
        \item \champ{Classe}
        \item \champ{Semestre}
    \end{itemize}
    \item Le système filtre automatiquement :
    \begin{itemize}
        \item Seuls les étudiants avec moyenne $\geq$ 10/20
        \item Tri par ordre alphabétique ou par mérite
    \end{itemize}
    \item \bouton{Générer les attestations}
\end{enumerate}

\subsubsection{Contenu de l'attestation}

\begin{itemize}[leftmargin=*]
    \item Titre : "ATTESTATION DE RÉUSSITE"
    \item Corps :
    \begin{itemize}
        \item Texte légal : "Le Doyen de la Faculté... certifie que..."
        \item Nom complet de l'étudiant
        \item Classe et année académique
        \item Mention obtenue
        \item Date de délivrance
    \end{itemize}
    \item Signatures : Doyen, Secrétaire
    \item Cachet de l'établissement
\end{itemize}

\subsection{Procès-verbaux d'examens}

\subsubsection{Génération de PV}

\begin{enumerate}[leftmargin=*]
    \item Menu \menu{Exports} > \menu{PV d'examens}
    \item Sélectionnez :
    \begin{itemize}
        \item \champ{Cours}
        \item \champ{Type d'examen}
        \item \champ{Session}
    \end{itemize}
    \item \bouton{Générer le PV}
\end{enumerate}

\subsubsection{Contenu du PV}

\begin{itemize}[leftmargin=*]
    \item En-tête officiel
    \item Informations de l'examen :
    \begin{itemize}
        \item Cours, type, date
        \item Enseignant responsable
        \item Nombre d'inscrits
    \end{itemize}
    \item Tableau des résultats :
    \begin{itemize}
        \item Liste nominative des étudiants
        \item Notes obtenues
        \item Mentions
    \end{itemize}
    \item Statistiques :
    \begin{itemize}
        \item Taux de réussite
        \item Moyenne de classe
        \item Répartition des mentions
    \end{itemize}
    \item Signatures : Enseignant, Jury, Direction
\end{itemize}

\subsection{Export en masse}

\subsubsection{Export par classe}

Pour exporter tous les relevés d'une classe :
\begin{enumerate}[leftmargin=*]
    \item Sélectionnez la classe
    \item Choisissez le format de sortie :
    \begin{itemize}
        \item \textbf{ZIP} : Archive de PDFs individuels
        \item \textbf{PDF unique} : Tous les relevés dans un seul fichier
        \item \textbf{Excel} : Tableau récapitulatif
    \end{itemize}
    \item \bouton{Générer}
    \item Téléchargez l'archive générée
\end{enumerate}

\subsection{Gestion des logos et en-têtes}

\subsubsection{Configuration des logos}

\begin{enumerate}[leftmargin=*]
    \item Préparez vos logos :
    \begin{itemize}
        \item Format : PNG avec transparence
        \item Taille recommandée : 800x200 px
        \item Résolution : 150 DPI minimum
    \end{itemize}
    \item Placez-les dans \code{apps/server/storage/logos/}
    \item Nommage :
    \begin{itemize}
        \item \code{logo-fmsp.png} : Logo principal
        \item \code{logo-ud.png} : Logo université (si applicable)
    \end{itemize}
    \item Redémarrez le serveur pour charger les nouveaux logos
\end{enumerate}

\subsubsection{Personnalisation des en-têtes}

Les en-têtes sont configurés via le template :
\begin{lstlisting}[language={}, frame=single, basicstyle=\small\ttfamily]
<header>
  <img src="{{logoUrl}}" alt="Logo">
  <h2>{{facultyName}}</h2>
  <p>{{address}}</p>
</header>
\end{lstlisting}

\subsection{Bonnes pratiques}

\begin{tcolorbox}[colback=green!5!white,colframe=green!75!black,title=Recommandations pour les exports]

\textbf{Préparation :}
\begin{itemize}[leftmargin=*]
    \item Vérifiez que toutes les notes sont validées
    \item Testez avec un ou deux étudiants avant l'export en masse
    \item Préparez des templates propres et professionnels
\end{itemize}

\textbf{Qualité :}
\begin{itemize}[leftmargin=*]
    \item Utilisez des logos haute résolution
    \item Vérifiez l'orthographe et la grammaire
    \item Respectez la charte graphique de l'établissement
\end{itemize}

\textbf{Sécurité :}
\begin{itemize}[leftmargin=*]
    \item Stockez les exports de manière sécurisée
    \item Ne distribuez que les documents finaux validés
    \item Conservez une archive des exports pour traçabilité
\end{itemize}
\end{tcolorbox}

\newpage
